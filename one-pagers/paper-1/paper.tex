\documentclass[11pt]{report}
\usepackage[dvips=true,bookmarks=true]{hyperref} % ... only needed for PDF generation
\usepackage{fancyhdr}
\usepackage{fullpage}
\pagenumbering{gobble}
\pagestyle{fancy}
\rfoot{David Carney (ID: 998752005)}
\lfoot{ECE 1774}

\usepackage[margin=0.80in]{geometry}

%\chead{}
%\lhead{ECE 1774}
%\rhead{Page \thepage}

%\lfoot{}
%\cfoot{}
%\rfoot{}

%\renewcommand{\headrulewidth}{0.4pt}
%\renewcommand{\footrulewidth}{0.4pt}

\begin{document}

\noindent\textbf{Locality Aware Dynamic Load Management for Massively Multiplayer Games}\newline
J. Chen, B. Wu, M. Delap, B. Knutsson, H.Lu, C. Amza
\newline
\noindent\textbf{Summarized By:} David Carney (ID: 998752005) [Sept 22, 2014]\newline

This paper explores ``dynamic load management through adaptive region to server
remapping'' of (massive) multiplayer games. In particular, it evaluates load
shedding and aggregation solutions in light of two competing goals: {\textit
balancing server load} in light of dynamic player positions; and {\textit
decreasing inter-server communication}. While dynamic load management is not
new, exploring the problem as applied to multiplayer games is, especially
in WAN environments. The paper presents and evaluates several algorithms,
backed by experimental and simulated results.
\newline
\newline
\noindent Strengths of the paper include:
\begin{itemize}
\item Generally concise, well-written, and unambiguous.
\item Comparison of LAN and WAN results offers good insights.
\item Comparison and analysis of various partitioning algorithms is straightforward and enlightening.
\end{itemize}

\noindent Weaknesses of the paper include:
\begin{itemize}
\item Simulation methodology is perhaps too simplistic and fails to address issues/problems characteristic of larger computer networks.
\item Makes various unstanted, but important assumptions (ex. players do not cheat, players do not collide). Such assumptions are invalid in commercial games.
\item Paper does not thoroughly discuss the real-time (i.e. instantanous and short-term) effects of region remapping on end-user experience.
\item The regions themselves are fixed-size, which does not fully alleviate problems related to flocking. Some discussion of dynamic region resizing would have been nice.
\end{itemize}

\end{document}

